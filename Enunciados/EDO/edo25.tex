\linespread{1.5}
\textbf{Reações Irreversíveis Mononucleares.} A lei de ação de massa estabelece que a velocidade de uma reação química é proporcional às concentrações das substâncias que reagem. Consideremos inicialmente a inversão da sacarose. A reação é dada por:
\begin{equation*}
    C_{12}H_{22}O_{11} + H_2O \rightarrow C_6H_{12}O_6 + C_6H_{12}O_6
\end{equation*}

Da reação, são formadas duas moléculas, uma de glicose e outra de frutose. Como, neste caso, a concentração da água pode ser suposta constante durante a reação, já que sua variação é desprezível nas condições em que o problema se realiza, chamaremos \textbf{\textit{A}} tal concentração. A concentração da sacarose antes de iniciar a reação será dada por \textbf{\textit{a}} e da sacarose decomposta ao fim do tempo \textbf{\textit{t}} de \textbf{\textit{x}}. A velocidade com que se verifica a inversão será dada pela derivada da quantidade decomposta em relação ao tempo. Como essa derivada deve ser proporcional às concentrações \textbf{\textit{A}} da água e \textbf{\textit{a-x}} da sacarose que ainda não reagiu, temos a seguinte equação:
\begin{equation}
    \label{eq:edo25}
    \frac{dx}{dt} = kA(a-x)
\end{equation}
Dessa forma, pede-se:
\begin{itemize}
    \item[\textbf{a)}] Ache a solução geral da E.D.O (\ref{eq:edo25});
    \item[\textbf{b)}] Esboce o campo de direções da E.D.O (\ref{eq:edo25}).
\end{itemize}
