\linespread{1.5}

\textit{\textbf{Solução}}

Dada a equação diferencial temos que, tomando a solução do tipo $ y(t) = Ce^pt $ podemos obter que o polinômio característico é dado por:
\begin{equation*}
    p^2 + p + 1 = 0 \Longleftrightarrow p = \frac{-1}{2}\pm i\frac{\sqrt{3}}{2}
\end{equation*}

dessa forma as soluções são:
\begin{equation*}
    \begin{cases}
    y_1 = C_1e^{\frac{-1}{2} + i\frac{\sqrt{3}}{2}}\\
    y_2 = C_2e^{\frac{-1}{2} - i\frac{\sqrt{3}}{2}}
    \end{cases}
\end{equation*}

E pelo princípio da superposição é dado que a solução geral é:
\begin{equation*}
    y_G(t) = C_1e^{\frac{-1}{2} + i\frac{\sqrt{3}}{2}} + C_2e^{\frac{-1}{2} - i\frac{\sqrt{3}}{2}}
\end{equation*}

Se supormos que $C_1 = C_2 = 1/2$:
\begin{equation*}
    y(t) = e^{\frac{-t}{2}}\left(\frac{e^{i\frac{\sqrt{3}}{2}} + e^{-i\frac{\sqrt{3}}{2}}}{2}\right) = e^{\frac{-t}{2}}cos\left(\frac{\sqrt{3}}{2}t\right)
\end{equation*}
Se supormos que $C_1 = 1/2i$ e $C_2 = -1/2i$:
\begin{equation*}
    y(t) = e^{\frac{-t}{2}}\left(\frac{e^{i\frac{\sqrt{3}}{2}} - e^{-i\frac{\sqrt{3}}{2}}}{2i}\right) = e^{\frac{-t}{2}}sin\left(\frac{\sqrt{3}}{2}t\right)
\end{equation*}

Combinando as soluções particulares e as generalizando, temos que a solução geral também pode ser dada por 
\begin{equation*}
    y_G = e^{-t/2}\left(D_1cos\left(\frac{\sqrt{3}}{2}t\right) + D_2sin\left(\frac{\sqrt{3}}{2}t\right)\right)
\end{equation*}