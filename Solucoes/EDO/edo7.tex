\linespread{1.5}

\textbf{Solução}

\textbf{a)} Dada a EDO $y'+2sin(2\pi x)=0$ e sabendo que sua solução é por integração, teremos que esta é separável então:
\begin{equation*}
    \frac{dy}{dx} = -2sin(2\pi x)
\end{equation*}
\begin{equation*}
    \int dy = -2\int sin(2\pi x) dx 
\end{equation*}
\begin{equation*}
    y + c_1 = -\cancel{2}\left(\frac{-cos(2\pi x)}{\cancel{2}\pi}\right) + c_2
\end{equation*}
tendo $C = c_2-c_1$
\begin{equation*}
    \boxed{y = \frac{cos(2x\pi)}{\pi} + C}
\end{equation*}

\textbf{b)}
Dada a EDO $y'+xe^a = 0$, onde $a=\frac{-x^2}{2}$, e sabendo que sua solução é por integração, teremos que esta é separável então:
\begin{equation*}
    \frac{dy}{dx} = -xe^a 
\end{equation*}
\begin{equation*}
    \int dy = \int -xe^a dx
\end{equation*}
Sabendo que $a = \frac{-x^2}{2}$, resolvemos a integral do termo da esquerda por substituição, onde
\begin{equation*}
    \begin{cases}
    u = a = \frac{-x^2}{2}\\
    du = -x dx
    \end{cases}
\end{equation*}
Assim
\begin{equation*}
    y + c_1 = \int e^u du = e^u + c_2
\end{equation*}
Sabendo que $C = c_2-c_1$ e substituindo $u = \frac{-x^2}{2}$, então:
\begin{equation*}
    \boxed{y = e^{-\frac{x^2}{2}} + C}
\end{equation*}

\textbf{c)} Dada a EDO $y' = y$ e sabendo que sua solução é por integração, teremos que esta é separável então:
\begin{equation*}
    \frac{dy}{dx} = y
\end{equation*}
Assim:
\begin{equation*}
    \int \frac{1}{y}dy = \int dx
\end{equation*}
\begin{equation*}
    ln(y) + c_1 = x + c_2
\end{equation*}
Sabendo que $C = c_2-c_1$, teremos:
\begin{equation*}
    ln(y) = x + C
\end{equation*}
aplicando a propriedade de logaritmo que $e^{ln(x)} = x$, e tirando o exponencial de ambos os lados, teremos que:
\begin{equation*}
    y = e^{x}e^C
\end{equation*}
sendo $e^C=D$ uma constante qualquer:
\begin{equation*}
    \boxed{y = De^x}
\end{equation*}

\textbf{d)} Dada a EDO $y' = 4e^{-x}cos(x)$ e sabendo que sua solução é por integração, teremos que esta é separável então:
\begin{equation*}
    \int dy = 4\int e^{-x}cos(x)
\end{equation*}
A integral da esquerda pode ser resolvida por partes, mas por experiencia sabemos que tal procedimento se repetiria varias vezes, então usaremos a abordagem que transformamos:
\begin{equation*}
    cos(x) = \frac{e^{ix} + e^{-ix}}{2}
\end{equation*}
\begin{equation*}
    y + c_1= 2\left[\int e^{(i-1)x}dx + \int e^-(i+1)x dx \right]
\end{equation*}
por questões de praticidade chamemos:
\begin{equation*}
    \begin{cases}
    a = (i-1)\\
    b = (i+1)
    \end{cases}
\end{equation*}
\begin{equation*}
    y = 2\left[\frac{e^{ax}}{a} - \frac{e^{-bx}}{b}\right] + c_2 - c_1
\end{equation*}
tomemos $c_2 - c_1 = C$, constantes arbitrárias:
\begin{equation*}
    y = 2\left[\frac{be^{ax} - be^{-bx}}{ab}\right] + C    
\end{equation*}
e temos que:
\begin{equation*}
    ab = (i-1)(1+i) = i + i^2 -1 - i = -1 -1 = -2 
\end{equation*}
\begin{equation*}
    y = -2e^{-x}\left[\frac{(1+i)e^{ix}-(i-1)e^{-ix}}{2}\right] + C = -2e^{-x}\left[\left(\frac{e^{ix}+e^{-ix}}{2}\right) - \left(\frac{e^{ix}-e^{-ix}}{2i}\right)\right]+C
\end{equation*}
Agora, nós sabemos que:
\begin{equation*}
    \begin{cases}
        cos(x) = \frac{e^{ix} + e^{-ix}}{2}\\
        sin(x) = \frac{e^{ix} - e^{-ix}}{2i}
    \end{cases}
\end{equation*}
e então substituirmos:
\begin{equation*}
    \boxed{y = 2e^{-x}\left(sin(x) - cos(x)\right)+C}
\end{equation*}

\textbf{e)} Dada a EDO $y' = -y$ e sabendo que sua solução é por integração, teremos que esta é separável então:
\begin{equation*}
    \frac{dy}{dx} = -y
\end{equation*}
Assim:
\begin{equation*}
    \int \frac{1}{y}dy = -\int dx
\end{equation*}
\begin{equation*}
    ln(y) + c_1 = -x + c_2
\end{equation*}
Sabendo que $C = c_2-c_1$, teremos:
\begin{equation*}
    ln(y) = -x + C
\end{equation*}
aplicando a propriedade de logaritmo que $e^{ln(x)} = x$, e tirando o exponencial de ambos os lados, teremos que:
\begin{equation*}
    y = e^{-x}e^C
\end{equation*}
sendo $e^C=D$ uma constante qualquer:
\begin{equation*}
    \boxed{y = De^{-x}}
\end{equation*}