\linespread{1.5}

\textbf{Solução}

\textbf{a)} As fórmulas consideradas gerais são as 3 primeiras (I, II, III).

\textbf{b)} Como consequência todas as outras fórmulas (IV, V, VI) são particulares, sendo que a IV e V são particulares quando a $f(t)$ é uma função par, devido ao fato que, na equação V, o produto de duas funções pares também é uma função par. 
O caso do item VI é particular para quando a $f(t)$ é uma função ímpar, e voltamos novamente a propriedade que o produto de funções impares é uma função par, permitindo essa transformação.