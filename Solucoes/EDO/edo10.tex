\linespread{1.5}

\textbf{Solução}

\textbf{a)}

Aqui é uma EDO de primeira ordem não homogênea. Para solucionarmos procuramos um fator integrante de forma que:

\begin{equation*}
    f(x)y'(x) + f(x)p(x)y(x) = f(x)r(x) 
\end{equation*}

onde a solução da EDO é dada por:
\begin{equation*}
    y_G = \frac{1}{f(x)}\left(\int r(x)f(x) dx + C \right)
\end{equation*}

e \begin{equation*}
    f(x) = e^{\int p(x) dx}
\end{equation*}

Nesse caso $p(x) = tan(x)$ e $r(x) = sin(2x)$

então:
\begin{equation*}
    f(x) = e^{\int tan(x) dx} = e^{-ln(cos(x))} = sec(x)
\end{equation*}

assim;

\begin{equation*}
    \int sin(2x)sec(x) dx = -2cos(x)
\end{equation*}

Dessa forma teremos que:

\begin{equation*}
    y_G = \frac{1}{sec(x)}(-2cos(x) + C) = -2cos^2(x) + Ccos(x)
\end{equation*}
\begin{equation*}
    \boxed{y_G = -2cos^2(x) + Ccos(x)}
\end{equation*}

\textbf{b)} Se y(0) = 1, então teremos que:

\begin{equation*}
    -2cos^2(0) + Ccos(0) = 1 \rightarrow -2 + C = 1 \rightarrow C = 3
\end{equation*}

Dessa forma a solução particular quando y(0) = 1 é:
\begin{equation*}
    \boxed{y_P = cos(x)(-2cos(x) + 3)}
\end{equation*}