\linespread{1.5}

\textbf{Solução}

Para verificar a paridade da função, basta determinarmos quais são os valores dos coeficientes reais de Fourier, pois se a função for par, $a_0=a_k=0\forall k \in \N$. No caso dela ser ímpar teríamos que $b_k=0\forall k \in \N$. Para testar isso, precisamos lembrar das relações de equivalência entre a forma real e complexa da série de Fourier:
\begin{equation*}
    \begin{cases}
        a_0 = c_0\\
        a_k = (c_k + c_{-k})\hspace{0.3cm}para\hspace{0.3cm}k\in\N\\
        b_k = i(c_k - c_{-k})\hspace{0.3cm}para\hspace{0.3cm}k\in\N
    \end{cases}
\end{equation*}

De cara já podemos ver que a função não é par, pois $a_0\neq0$, então basta verificarmos $b_k$:
\begin{equation*}
    b_k = i\left[\frac{1+ik}{k^4} - \left(\frac{1-ik}{(-k)^4}\right) \right] = i\left(\frac{1 -1 + ik + ik}{k^4}\right) = \frac{2i^2k}{k^4} = \frac{-2}{k^3} \neq 0 \forall k \in \N
\end{equation*}

Dessa forma mostramos que nenhum dos coeficientes de Fourier reais é nulo, dessa forma podemos afirmar que a função representada não é nem par nem ímpar.