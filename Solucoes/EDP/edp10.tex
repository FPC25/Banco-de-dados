\linespread{1.5}

\textbf{Solução}

\textbf{a)}

substituímos $cos(\theta) = s$ e então calculamos os valor de $P_3(s)$:

\begin{equation*}
    P_3(s) = \frac{1}{3!2^3}\frac{d^3}{ds^3}(s^2-1)^3
\end{equation*}

calculemos as derivadas:
\begin{equation*}
    \frac{d}{ds} (s^2-1)^3 = 3(s^2-1)^22s = 6s(s^2-1)^2
\end{equation*}
\begin{equation*}
    \frac{d^2}{ds^2} (s^2-1)^3 = 6[(s^2-1)^2 + 4s^2(s^2-1)] = 6(s^2-1)^2 + 24s^2(s^2-1)
\end{equation*}
\begin{equation*}
    \frac{d^3}{ds^3}(s^2-1)^3 = 24s(s^2-1) + 48s(s^2-1) + 48s^3 = 72s(s^2-1) + 48s^3 = 120s^3 - 72s
\end{equation*}

Voltando para o polinômio e substituindo $s=cos(\theta)$, obtemos:
\begin{equation*}
    P_3(cos(\theta)) = \frac{1}{48}(120cos^3(\theta) - 72cos(\theta)) = \frac{5}{2}cos^3(\theta) - \frac{3}{2}cos(\theta)
\end{equation*}

Dessa forma a solução axi-simétrica em coordenadas esféricas é:
\begin{equation*}
    \boxed{r^3\left(\frac{5}{2}cos^3(\theta) - \frac{3}{2}cos(\theta)\right)}
\end{equation*}

\textbf{b)}

Como temos coord. esféricas, temos que:
\begin{equation*}
    \begin{cases}
    r = \sqrt{x^2 + y^2 + z^2}\\
    cos(\theta) = \frac{x}{\sqrt{x^2 + y^2 + z^2}}
    \end{cases}
\end{equation*}
e então:
\begin{equation*}
    u(x,y,z) = (x^2 + y^2 + z^2)\sqrt{x^2 + y^2 + z^2}\left(\frac{5}{2}\frac{x^3}{(x^2 + y^2 + z^2)\sqrt{x^2 + y^2 + z^2}} - \frac{3}{2}\frac{x}{\sqrt{x^2 + y^2 + z^2}}\right)
\end{equation*}
\begin{equation*}
    u(x,y,z) = \frac{5}{2}x^3 - \frac{3}{2}x^3 - \frac{3}{2}xy^2 - \frac{3}{2}xz^2 \Rightarrow \boxed{u(x,y,z) = x^3 - \frac{3}{2}x(y^2 - z^2)}
\end{equation*}

Essa solução também é solução da eq. de Laplace nas coord. cartesianas:
\begin{equation*}
    \begin{cases}
        u_{xx} = 6x\\
        u_{yy} = -3x\\
        u_{zz} = -3x\\
        \Rightarrow \nabla^2 u(x,y,z) = u_{xx} + u_{yy} + u_{zz} = 0
    \end{cases}
\end{equation*}

Assim botamos que u(x,y,z) é composto de monômios de mesma ordem podemos dizer que é um polinômio homogêneo, e também é solução da EDP de Laplace.


