\linespread{1.5}

\textbf{Solução}

\textbf{a)} 

O método de separação de variáveis é dado por:
\begin{equation*}
    u(x,t) = X(x)\cdot T(t)
\end{equation*}
que satisfaz:
\begin{equation*}
    \begin{cases}
        i)\hspace{0.2cm} u_x (0, t) = X'(0)\cdot T(t) = 0\\
        ii)\hspace{0.2cm} u_x (L, t) = X'(L)\cdot T(t) = 0
    \end{cases} \forall t \in \Z
\end{equation*}

de \textit{i)} temos que \begin{equation*}
    \begin{cases}
        X'(0) = 0\\
            ou\\
        T(t) = 0
    \end{cases}
\end{equation*}

e de \textit{ii)} temos que \begin{equation*}
    \begin{cases}
        X'(L) = 0\\
            ou\\
        T(t) = 0
    \end{cases}
\end{equation*}

Em ambos os casos, sendo $T(t) =0 $ teremos solução trivial uma vez que $u(x,0) = X(x)\cdot 0 = 0$ e solução trivial não nos é interessante. Então estudaremos as outras opções que nos dão as condições de contorno:
\begin{equation*}
    \begin{cases}
         X'(0) = 0\\
         X'(L) = 0
    \end{cases}
\end{equation*}

Impondo que:
\begin{equation*}
    \begin{cases}
         u_t = X(x)T'(t)\\
         u_{tt} = X(x)T''(t)\\
         u_xx = X''(x)T(t) 
    \end{cases}
\end{equation*}
e substituindo na equação de onda teremos que:
\begin{equation*}
     X''(x)T(t) = a^{-2}X(x)T''(t) \longrightarrow \frac{X''(x)}{X(x)} = a^{-2}\frac{T''(t)}{T(t)} = C
\end{equation*}
Onde $C$ é uma constante.

Assim, buscaremos resolver a edp com as soluções não triviais:
\begin{equation*}
    \frac{X''}{X} = C \longrightarrow X'' - CX = 0
\end{equation*}

uma EDO linear homogênea de segunda ordem, onde a solução vai diferir do valor que $C$ assume, se $C=0$, $C<0$ ou $C>0$. 

\begin{enumerate}[I]
    \item) $C = 0$:
    Se $X''(x) = 0 \rightarrow X'(x) = m$, sendo $m$ uma constante e assim $X(x) = mx+b$ e se aplicarmos as condições de contorno teremos que:
    \begin{equation*}
        \begin{cases}
             X'(0) = 0 = m\\
             X'(L) = 0 = m
        \end{cases}
    \end{equation*}
    assim, $\boxed{X(x) = b}$, sendo que $b$ é uma constante qualquer que seja x, e então $u(x,t) = bT(t)$.
    \item) $C > 0$:
    Para fins de simplicidade chamaremos $C=\lambda^2$ e a solução desta EDO é do tipo $X(x) = e^{px}$. Então:
    \begin{equation*}
        e^{px}(p^2-\lambda^2) = 0
    \end{equation*}
    como $e^{px} \neq 0$, então $p^2 - \lambda^2 = 0$ é polinômio característico, e $p = \pm \lambda$.
    
    Dessa forma a EDO tem solução:
    \begin{equation*}
        \begin{cases}
            X(x) = \alpha e^{\lambda x} + \beta e^{-\lambda x}\\
            X'(x) = \lambda(\alpha e^{\lambda x} - \beta e^{-\lambda x})
        \end{cases}
    \end{equation*}
    sendo $\alpha, \beta$ constantes. Aplicando as condições de contorno teremos que, quando $x=0$:
    \begin{equation*}
        \lambda(\alpha - \beta) = 0
    \end{equation*}
    Como $\lambda = \sqrt{C} > 0$, então $\alpha = \beta$. Com $x = L$, teremos:
    \begin{equation*}
        \alpha \lambda (e^{\lambda L} - e^{-\lambda L}) = 0 
    \end{equation*}
    Novamente teremos que $lambda > 0$, $e^{\lambda L} - e^{-\lambda L} \neq 0$ pois $\lambda, L > 0$, então $\alpha = \beta = 0$. Com isso obtemos que $X(x) \equiv 0$ e portanto temos novamente a solução trivial que não nos interessa.
    \item) $C < 0$:
    
    Para fins de simplicidade tomaremos que $C = -\lambda^2$, sendo $\lambda \in \R^*$. Assim temos o polinômio característico:
    \begin{equation*}
        p^2 + \lambda^2 = 0 \longleftrightarrow p \pm i\lambda
    \end{equation*}
    Nos utilizando do princípio da sobreposição e conhecimento prévio temos que:
    \begin{equation*}
        \begin{cases}
            X(x) = \alpha\cos(\lambda x) + \beta\sin(\lambda x)\\
            X'(x) = \lambda(-\alpha\sin(\lambda x) + \beta\cos(\lambda x))
        \end{cases}
    \end{equation*}
    Aplicando as condições de contorno e lembrando que $\lambda \neq 0$:
    \begin{equation*}
        \begin{cases}
            \lambda(\beta \cos(0) - \alpha \sin(0)) = \lambda(\beta 1 - \alpha 0) = \lambda \beta = 0 \rightarrow \beta = 0\\
            -\lambda \alpha \sin(\lambda L) = 0
        \end{cases}
    \end{equation*}
    
    Nesse ultimo caso teremos que:
    \begin{equation*}
        ou \begin{cases}
           \alpha = 0 \leftarrow Solução \hspace{0.2cm} trivial\\
           \sin(\lambda L) = 0 \rightarrow \lambda = \frac{k\pi}{L}, k\in\Z \neq 0 
        \end{cases}
    \end{equation*}
    Descartando todas as soluções triviais, teremos que: 
    \begin{equation*}
        \boxed{X(x) = \alpha cos\left(x\frac{k\pi}{L}\right)}
    \end{equation*}
    Soluções que satisfazem as condições de contorno.
\end{enumerate}

Obtivemos duas soluções não triviais de $X(x)$ tanto para $C=0$, quanto para $C < 0$, e então solucionaremos:
\begin{equation*}
    \frac{1}{a^2}\frac{T''(t)}{T(t)} = C
\end{equation*}
para os $C$s já mencionados.

\begin{enumerate}[I]
    \item) $C = 0$:
    Se $T'' = 0 \Longleftrightarrow T(t) = gt + f$, sendo $g, f$ constantes e assim $u(x, t) = b(gt + f) = c_1t + c_2$, constantes arbitrárias do produto entre $b, g$ e $f$.
    
    Contudo a solução encontrada para essa condição de $C=0$ não representa uma onda de função periódica e portanto é uma solução fisicamente absurda e a desconsideramos.
    \item) $C < 0$:
    
    novamente tomamos $C = -\lambda^2$ e sabemos que $\lambda = \frac{k\pi}{L}$:
    \begin{equation*}
        T'' + (a\lambda)^2T = 0
    \end{equation*}
    Sabemos que a solução dessa EDO é do tipo $T(t) = e^{pt}$ e seu polinômio característico é:
    \begin{equation*}
        p^2 + (a\lambda)^2 = 0 \longrightarrow p \pm i\frac{ak\pi}{L}
    \end{equation*}
    Com conhecimento prévio e pelo princípio da superposição temos que a solução geral da EDO é:
    \begin{equation*}
        T(t) = Acos\left(\frac{ak\pi t}{L} + Bsin(\frac{ak\pi t}{L}\right)
    \end{equation*}
    Com $A,B\in\R$
    
    E dessa forma a solução da EDP é dada por:
    \begin{equation*}
        u(x, t) = \alpha cos\left(\frac{k\pi x}{L}\right)\left(Acos\left(\frac{ak\pi t}{L}\right) + Bsin\left(\frac{ak\pi t}{L}\right)\right)
    \end{equation*}
\end{enumerate}

Como vimos a única solução viável e que fisicamente faz sentido é quando $C<0$, mas podemos considerar $\omega = \frac{\pi}{L}$, $p = A\alpha$ e $q = B\alpha$, teremos o conjunto de soluções particulares:
\begin{equation*}
    u_k(x, t) = cos(k\w x)[pcos(a\w kt) + qsin(ak\w t)]
\end{equation*}

e quando aplicamos o princípio da superposição:

\begin{equation*}
    u(x, t) = \sum_{k=1}^\infty cos(k\w x)[p_kcos(a\w kt) + q_ksin(ak\w t)]
\end{equation*}
com $k\neq0\in\Z$  e $p_k < q_k$. 

\vspace{1cm}
\textbf{b)}

A função é periódica se $\exists P>0$ tal que $u(x, t) = u(x, t+P)$. Aplicando isso teremos que:
\begin{equation*}
    u(x, t+P) = cos(k\w x)[p_kcos(a\w k(t+P))+q_ksin(a\w k(t+P))]
\end{equation*}
chamemos $n=ak\w$ para fins de simplicidade:
\begin{equation*}
    = cos(kx\w)[p_k(cos(nt)cos(nP)-sin(nt)sin(nP)) + q_k(sin(nt)cos(nP) + sin(nP)cos(nt))]
\end{equation*}
\begin{equation*}
    = cos(kx\w)[cos(nP)(p_kcos(nt) + q_ksen(nt)) + sin(nP)(q_kcos(nt) - p_ksin(nt))]
\end{equation*}

Abrindo $n$ e igualarmos isso a frequência angular teremos que:
\begin{equation*}
    n =  \frac{ka\pi}{L} = \frac{2\pi}{P} \longrightarrow \boxed{P=\frac{2L}{ka}}
\end{equation*}

e então substituirmos isso na equação achada para $u(x, t+P)$:
\begin{equation*}
    u(x, t+P) = cos(k\w x)\left\{cos\left(\frac{\cancel{ak}\pi}{\cancel{L}}\frac{2\cancel{L}}{\cancel{ka}}\right)[p_kcos(nt)+q_ksin(nt)] + sin\left(\frac{\cancel{ak}\pi}{\cancel{L}}\frac{2\cancel{L}}{\cancel{ka}}\right)[q_kcos(nt) - p_ksin(nt)] \right\}
\end{equation*}
\begin{equation*}
    u(x, t+P) = cos(kx\w)[cos(2\pi)(p_kcos(nt) + q_ksen(nt)) + sin(2\pi)(q_kcos(nt) - p_ksin(nt))]
\end{equation*}
Sabendo que:
\begin{equation*}
    \begin{cases}
        cos(2\pi) = 1\\
        sin(2\pi) = 0
    \end{cases}    
\end{equation*}
obtemos que:
\begin{equation*}
    \boxed{u(x, t+P) = cos(kx\w)[p_kcos(nt) + q_ksen(nt)] = u(x, t)}
\end{equation*}
$\therefore u(x, t)$ é uma função periódica no tempo. 

Além disso, teremos que o maior período desta fução acontece quando $k=1$, pois:
\begin{equation*}
    P(k=1) = \frac{2L}{a}, \hspace{0.2cm} k\in\Z^*
\end{equation*}

\textbf{c)}

Com $L=50m$, $k=1$, $g=9,8ms^{-2}$ e $h=2m$:
\begin{equation*}
    a = \sqrt{hg} = \sqrt{2\cdot 9,8} = \sqrt{19,6}
\end{equation*}
Como $P$ tem que ser um valor positivo e $L$ é um valor positivo, então obrigatoriamente $a$ também deve ser um valor positivo. Dessa forma:
\begin{equation*}
    P = \frac{2L}{a} = \frac{100}{\sqrt{19,6}} \approx 22,6s
\end{equation*}