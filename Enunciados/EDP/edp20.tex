\linespread{1.5}

Uma haste longa e fina de cobre está inicialmente à temperatura de $0ºC$. No instante $t=0$, a extremidade da haste é colocada em contato com um ferro se solda à temperatura de $250ºC$. Sabendo que no instante $t$ a temperatura da haste em um ponto, situado à distância $x$ da extremidade, é dada pela função:
\begin{equation*}
    u(x,t) = u_0\left[1-erf\left(\frac{x}{2a\sqrt{t}}\right)\right]
\end{equation*}
sendo $u_0 = 250ºC$ e que a difusividade térmica do cobre é $a^2 = 1.11cm^2/s$, pede-se:
\begin{itemize}
    \item[a)]Calcule o valor de $t$ para o qual a temperatura de um ponto, situado à distância $x=2$ cm da extremidade, atinge o valor de $100ºC$
    \item[b)] Refaça o calculo do item a) para o caso de um haste de vidro, para qual $a^2 = 0.008 cm^2/s$
\end{itemize}