\linespread{1.5}

\textit{\textbf{Solução}}
Como é uma E.D.O. de segunda ordem de coeficientes constantes não homogênea  teremos que utilizar o método de Lagrange. Sabemos que:
\begin{equation*}
    y_G(x) = y_H(x) + y_P(x)
\end{equation*}
onde 
\begin{itemize}
    \item $y_G(t)$ é a solução geral da equação;
    \item $y_H(t)$ é a solução da homogênea ;
    \item $y_P(t)$ é a solução particular.
\end{itemize}

Resolvemos então primeiramente a homogênea. Dada a equação diferencial temos que, tomando a solução do tipo $ y(t) = Ce^pt $ podemos obter que o polinômio característico é dado por:
\begin{equation*}
    p^2 - 1 = 0 \Longleftrightarrow p = \pm 1
\end{equation*}

dessa forma as soluções são:
\begin{equation*}
    \begin{cases}
    y_1 = C_1e^x\\
    y_2 = C_2e^{-x}
    \end{cases}
\end{equation*}

E pelo princípio da superposição é dado que a solução geral é:
\begin{equation*}
    y_H(t) = C_1e^x + C_2e^{-x}
\end{equation*}

Agora voltemos buscamos a solução da particular pelo método de Lagrange. Para isso descobrimos o determinante da matriz:
\begin{equation*}
    W = \begin{vmatrix} 
        y_1' & y_2'\\
        y_1  & y_2
        \end{vmatrix} = 
        \begin{vmatrix} 
        C_1e^{x} & -C_2e^{-x}\\
        C_1e^{x} &  C_2e^{-x}
        \end{vmatrix} = C_1C_2 - (-C_1C_2) = 2C_1C_2
\end{equation*}

e agora determinamos $y_P$:
\begin{equation*}
  y_P = y_1u_1 + y_2u_2 = y_1\int \frac{r(x)y_2}{W} dx + y_2\int \frac{-r(x)y_1}{W} dx = (I) + (II)
\end{equation*}
onde $r(x) = len(x)$. Resolvendo $(I)$:
\begin{equation*}
    (I) = y_1\int \frac{r(x)y_2}{W} dx = \cancel{C_1}e^x\int \frac{ln(x)\cancel{C_2}e^{-x}}{2\cancel{C_1}\cancel{C_2}} dx = e^x\left(\frac{-e^{-x}ln(x) - Ei(-x)}{2}\right) = \frac{-ln(x) - e^xEi(-x)}{2}
\end{equation*}

\begin{equation*}
    (II) = y_2\int -\frac{r(x)y_1}{W} dx = \cancel{C_2}e^{-x}\int -\frac{ln(x)\cancel{C_1}e^{x}}{2\cancel{C_1}\cancel{C_2}} dx = e^{-x}\left(\frac{-e^{x}ln(x) + Ei(x)}{2}\right) = \frac{-ln(x) + e^{-x}Ei(x)}{2}
\end{equation*}

onde $Ei(x) = \int \frac{e^x}{x} dx$, integral exponencial sem solução analítica. Assim a solução da particular é dada por:
\begin{equation*}
    y_P (x) = \frac{-2ln(x) - e^xEi(-x) + e^{-x}Ei(x)}{2}
\end{equation*}

Dessa forma a solução geral é dada por:
\begin{equation*}
    y_G (x) = C_1e^x + C_2e^{-x} + \frac{-2ln(x) - e^xEi(-x) + e^{-x}Ei(x)}{2}
\end{equation*}