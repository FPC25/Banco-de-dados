\linespread{1.5}
Seja $f(x)$ uma função periódica com periódo $T = 2\pi$, definida como:

\begin{equation*}
    f(x) = \begin{cases}
     1, \hspace{1cm} para\hspace{0.7cm} 0 \leq x < \pi;\\
     0, \hspace{1cm} para\hspace{0.2cm} -\pi \leq x < 0;
    \end{cases}\,
\end{equation*}

Nessas condições, pede-se:
\begin{itemize}
    \item[\textbf{a)}] Faço o esboço do gráfico de $f(x)$ no intervalo $-3\pi \leq x < 3\pi$;
    \item[\textbf{b)}] Calcule a série de Fourier de $f(x)$;
    \item[\textbf{c)}] Use os resultados obtidos nos itens \textbf{a)} e \textbf{b)} para obter uma expressão em série para $\pi$;
    \item[\textbf{d)}] Usando o resultado do item \textbf{c)}, calcule o valor de $\pi$ quando truncamos a série de Fourier com 2 termos e 3 termos. Avalie a diferença.
\end{itemize}