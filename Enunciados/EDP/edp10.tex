\linespread{1.5}

A equação de Laplace em coordenadas esféricas possui infinitas soluções axi-simétricas (ou seja, independentes de $\lambda$) dadas por $u_n(r, \theta) = r^nP_n(cos(\theta))$, com $n\in\Z$, sendo $P_n(cos(\theta))$ o polinômio de Legendre $P_n(s)$ de grau $n$, com $s = cos(\theta)$. Os polinômios de Legendre podem ser calculados pela fórmula de Rodrigues: $P_n(s) = \frac{1}{2^n n!}\frac{d^n}{ds^n}(s^2-1)^n$. Nesses condições, pede-se:
\begin{itemize}
    \item[a)] Escreva a expressão de $u_n(r, \theta)$, conforme acima definida, explicitamente em termos de \textit{r} e $\theta$ para $n = 3$.
    \item[b)] Mostre que $u_n (r, \theta)$ do item a), quando expressa em coordenadas cartesianas, tem a forma de um polinômio homogêneo, o qual é solução da equação de Laplace em coordenadas cartesianas.
\end{itemize}