\linespread{1.5}

\textbf{Solução}
%Integral
A integral é dada por 
\begin{equation*}
    f(x) = \int_0^\infty [A(w)cos(wx) + B(w)sin(wx)]dw
\end{equation*}
onde:
\begin{equation*}
    A(w) = \frac{1}{\pi}\int_{-\infty}^\infty f(v) cos(wv)dv
\end{equation*}
e
\begin{equation*}
    B(w) = \frac{1}{\pi} \int_{-\infty}^\infty f(v) sin(wv)dv
\end{equation*}

como para $x, k> 0$ a função $f(x)$ é absolutamente convergente, já que $\int_0^\infty |e^{-kx}|dx = \int_0^\infty e^{-kx}dx = \frac{1}{k}$ e $f(x)$ é seccionalmente contínua, pelo teorema de Fouerier a integral de Fourier existe. Logo:
\begin{equation*}
    A(w) = \frac{1}{\pi}\int_0^\infty e^{-kv} cos(wv)dv = \frac{1}{\pi} (I)
\end{equation*}
\begin{equation*}
    (I) = \left\{\left[\frac{cos(wv)e^{-kv}}{-k}\right]^\infty_0 - \int_0^\infty \frac{-wsin(wv)e^{-kv}}{-k}dv\right\} 
\end{equation*}
\begin{equation*}
    = \left\{\left[\frac{cos(wv)e^{-kv}}{-k}\right]^\infty_0 - \left[\frac{-wsin(wv)e^{-kv}}{k^2}\right]^\infty_0 + \int_0^\infty \frac{-w^2cos(wv)e^{-kv}}{k^2}dv \right\}
\end{equation*}
\begin{equation*}
     \int_0^\infty e^{-kv} cos(wv)dv  + \int_0^\infty \frac{w^2cos(wv)e^{-kv}}{k^2}dv = \left[\frac{-cos(wv)e^{-kv}}{k}\right]^\infty_0 + \left[\frac{wsin(wv)e^{-kv}}{k^2}\right]^\infty_0
\end{equation*}
\begin{equation*}
    \left(\frac{w^2 + k^2}{k^2}\right)\int_0^\infty e^{-kv} cos(wv)dv = \frac{1}{k^2}\left\{\left[-kcos(wv)e^{-kv}\right]^\infty_0 + \left[wsin(wv)e^{-kv}\right]^\infty_0\right\}
\end{equation*}
\begin{equation*}
    \int_0^\infty e^{-kv} cos(wv)dv = \frac{1}{w^2 + k^2}\left\{\left[-kcos(wv)e^{-kv}\right]^\infty_0 + \left[wsin(wv)e^{-kv}\right]^\infty_0\right\} = \frac{1}{w^2 + k^2}\{(II) + (III)\}
\end{equation*}

\begin{equation*}
    (II) = -k (\lim_{v\rightarrow \infty}-kcos(wv)e^{-kv} - 1)
\end{equation*}

Para resolver esse limite, utilizamos o \underline{teorema do confronto}. Como a função cosseno é limitad entre $-1$ e $1$, logo temos que:
\begin{equation*}
    -1\leq cos(wv) \leq 1
\end{equation*}

Multiplicando a expressão acima por $e^{-kv}$, obtemos:
\begin{equation*}
    -e^{-kv} \leq e^{-kv}cos(wv) \leq e^{-kv}
\end{equation*}
Tirando o limite da expressão acima, temos:
\begin{equation*}
    - \lim_{v\rightarrow \infty}e^{-kv} \leq  \lim_{v\rightarrow \infty}e^{-kv}cos(wv) \leq  \lim_{v\rightarrow \infty}e^{-kv}
\end{equation*}
\begin{equation*}
    0 \leq  \lim_{v\rightarrow \infty}e^{-kv}cos(wv) \leq  0
\end{equation*}

Portanto:
\begin{equation*}
    \boxed{\lim_{v\rightarrow \infty}e^{-kv}cos(wv) =  0}
\end{equation*}
O mesmo raciocínio se utiliza para (III) e temos que:
\begin{equation*}
    \boxed{\lim_{v\rightarrow \infty}we^{-kv}sen(wv) =  0}
\end{equation*}

\begin{equation*}
    \int_0^\infty e^{-kv} cos(wv)dv = \frac{1}{w^2 + k^2}k = \frac{k}{k^2+ w^2}
\end{equation*}
\begin{equation*}
    \boxed{A(w) = \frac{k}{\pi(k^2+w^2)}}
\end{equation*}

Por fim resolvemos $B(w)$
\begin{equation*}
    B(w) = \frac{1}{\pi}\int_0^\infty e^{-kv}sin(wv)dv = \frac{1}{\pi}(IV)
\end{equation*}

\begin{equation*}
    \int_0^\infty e^{-kv}sin(wv)dv = \left[\frac{sin(wv)e^{-kv}}{-k}\right]^\infty_0 - \int_0^\infty \frac{wcos(wv)e^{-kv}}{-k}dv
\end{equation*}
\begin{equation*}
    = \left[\frac{sin(wv)e^{-kv}}{-k}\right]^\infty_0 - \left\{\left[\frac{wcos(wv)e^{-kv}}{k^2}\right]^\infty_0 -\int_0^\infty \frac{-w^2sin(wv)e^{-kv}}{k^2}dv\right\}
\end{equation*}
\begin{equation*}
    = \left[\frac{sin(wv)e^{-kv}}{-k}\right]^\infty_0 -\left[\frac{wcos(wv)e^{-kv}}{k^2}\right]^\infty_0 - \int_0^\infty \frac{w^2sin(wv)e^{-kv}}{k^2}dv
\end{equation*}
\begin{equation*}
    \int_0^\infty e^{-kv}sin(wv)dv + \int_0^\infty \frac{w^2sin(wv)e^{-kv}}{k^2}dv = \left[\frac{sin(wv)e^{-kv}}{-k}\right]^\infty_0 -\left[\frac{wcos(wv)e^{-kv}}{k^2}\right]^\infty_0 
\end{equation*}
\begin{equation*}
    \left[\frac{k^2 + w^2}{k^2}\right]\int_0^\infty e^{-kv}sin(wv)dv =  \frac{1}{k^2}\left\{\left[-ksin(wv)e^{-kv}\right]^\infty_0 -\left[wcos(wv)e^{-kv}\right]^\infty_0 \right\}
\end{equation*}
\begin{equation*}
    \int_0^\infty e^{-kv}sin(wv)dv = \frac{1}{k^2 + w^2}   \left\{\left[-ksin(wv)e^{-kv}\right]^\infty_0 -\left[wcos(wv)e^{-kv}\right]^\infty_0 \right\} =  \frac{1}{k^2 + w^2}  \left\{(V) - (VI)\right\} 
\end{equation*}

\begin{equation*}
    (V) = \lim_{v\rightarrow \infty}-ke^{-kv}sen(wv) - 0 =  0
\end{equation*}

\begin{equation*}
    (VI) = \lim_{v\rightarrow \infty}we^{-kv}cos(wv) - w = w 
\end{equation*}

\begin{equation*}
    \int_0^\infty e^{-kv}sin(wv)dv = \frac{w}{k^2 + w^2}
\end{equation*}

\begin{equation*}
    \therefore B(w) = \frac{w}{\pi(k^2 + w^2)}
\end{equation*}

Dessa forma a integral de Fourier é dada por:
\begin{equation*}
    \boxed{f(x) = \frac{1}{\pi}\int_0^\infty \frac{kcos(wx) + wsin(wx)}{k^2+w^2}dw  } 
\end{equation*}