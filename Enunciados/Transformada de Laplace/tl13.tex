\linespread{1.5}
Seja $f(t)$ uma função real de variável real. Pede-se:
\begin{itemize}
    \item[\textbf{a)}] Quais dos itens abaixo são as condições suficiente para que $f(t)$ seja um original, isto é, para que admita transformada de Laplace:
    \begin{enumerate}[I]
        \item $f(t) = h(t), t>0$, onde $h(t)$ é a função de Heaviside;
        \item $f(t)$ seja seccionalmente contínua em qualquer intervalo finito;
        \item $f(t)$ não cresce mais rapidamente que um polinômio $t^T$, isto é, existem constantes positivas \textit{M} e \textit{r} tais que $|f(t)| \leq M\cdot t^T$;
        \item $f(t)$ deve ser absolutamente convergente em $-\infty < t < +\infty$, isto é, $\int_{-\infty}^{+\infty}|f(t)|dt < \infty$;
        \item $f(t) = 0$ para $t<0$;
        \item $f(t)$ não cresce mais rapidamente que uma exponencial, isto é, existem constantes positivas finitas \textit{M} e $\alpha$ tais que $|f(t)| \leq M\cdot e^{\alpha t}$;
        \item $f(t)$ não cresce mais rapidamente que uma exponencial, isto é, existem constantes positivas finitas \textit{M} e $\alpha$ tais que $|f(t)| \geq M\cdot e^{\alpha t}$;
        \item Os itens II, IV, V, VII são verdadeiros.
    \end{enumerate}
    \item[\textbf{b)}] Suponha que $f(t) = \cos{(t-2\pi)}$. Essa função é limitada e está definida em todo o intervalo $-\infty < t < +\infty$. No contexto da transformada de Laplace, como essa função deve ser escrita para ser considerada em original?
    \item[\textbf{c)}] Esboce o gráfico do original $f(t) = cos(t-2\pi)$ e ache sua transformada de Laplace.
\end{itemize}