\linespread{1.5}
Considere um modelo simplificado para o fenômeno da variação de temperatura num corpo, por perda de calor para o ambiente, fazendo as seguintes hipóteses:
\begin{enumerate}[I]
    \item A temperatura $T$ é amesma em todo o corpo e depende apenas do tempo $t$;
    \item A temperatura $T_a$ do ambiente é constante com o tempo e é a mesma em todo o ambiente;
    \item O fluxo de calor através das paredes do corpo é dado por $\frac{dT}{dt}$, é proporcional à diferença entre as temperaturas do corpo e do ambiente:
    \begin{equation}
        \frac{dT}{dt} = -k(T-T_a)
        \label{eq:edo331}
    \end{equation}
    sendo $k$ uma constante positiva que depende das propriedades físicas do corpo. O sinal negativo da equação (\ref{eq:edo331}) acima se explica pelo fato que o calor flui da fonte quente para a fonte fria.
\end{enumerate}
\begin{itemize}
    \item[\textbf{a)}] Determine a solução particular da equação \ref{eq:331}, dada a seguinte condição inicial:
    $T(0) = T_0$;
    \item[\textbf{b)}] Esboce o gráfico da solução.
\end{itemize}