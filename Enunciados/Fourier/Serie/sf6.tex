\linespread{1.5}
Seja $f(t)$ uma função periódica, tal que:
\begin{equation*}
    f(t) = t+\pi \hspace{1cm} para \hspace{0.2cm} -\pi \leq t < \pi \hspace{0.25cm} e  \hspace{0.25cm} \f(t+2\pi) = f(t)
\end{equation*}

Nessas condições, pergunta-se:
\begin{itemize}
    \item[\textbf{a)}] Qual é o período da função $f(t)$?
    \item[\textbf{b)}] Faça o esboço do gráfico de $f(t)$ no intervalo $-3\pi \leq t < 3\pi$;
    \item[\textbf{c)}] Desenvolva a série de Fourier de $f(t)$.
\end{itemize}