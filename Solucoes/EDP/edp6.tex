\linespread{1.5}

\textbf{Solução}

Aplicando a primeira condição de contorno:
\begin{equation*}
    u(0,t) = pcos(rt) = 0
\end{equation*}
Como $cos(rt) \neq 0$, pois caso contrário teríamos a solução trivial, obtemos que $\boxed{p=0}$. Aplicando a outra condição de contorno e já substituindo $p$:
\begin{equation*}
    u(10,t) = sin(10q)cos(rt) = 0 \Rightarrow sin(q10) = 0
\end{equation*}
\begin{equation*}
    \boxed{q = \frac{k\pi}{10}}, k\in\Z^*
\end{equation*}

Com isso obtemos já uma solução que satisfaz as condições de contorno:
\begin{equation*}
    u(x,t) = sin\left(\frac{k\pi x}{10}\right)cos(rt)
\end{equation*}
Mas agora precisamos determinar a solução particular, ou seja, determinar o valor de $r$:
\begin{equation*}
    \begin{cases}
        u_{tt} = -r^2sin\left(\frac{k\pi x}{10}\right)cos(rt)\\
        u_{xx} = -\left(\frac{k\pi}{10}\right)^2sin\left(\frac{k\pi x}{10}\right)cos(rt)
    \end{cases}
\end{equation*}
Substituindo na equação de onda:
\begin{equation*}
    -r^2\cancel{sin\left(\frac{k\pi x}{10}\right)cos(rt)} = -25\left(\frac{k\pi}{10}\right)^2\cancel{sin\left(\frac{k\pi x}{10}\right)cos(rt)}
\end{equation*}
\begin{equation*}
    r^2 = 25\left(\frac{k\pi}{10}\right)^2 \Rightarrow r = \pm 5\left(\frac{k\pi}{10}\right) = \pm \frac{k\pi}{2}
\end{equation*}
Sabendo que a função cosseno é uma função par, ou seja, $cos(x) = cos(-x)$:
\begin{equation*}
    \boxed{r = \frac{k\pi}{2}}
\end{equation*}
Dessa forma obtemos que a solução particular para essa EDP é dada por:
\begin{equation*}
    \boxed{u(x,t) = sin\left(\frac{k\pi x}{10}\right)cos\left(\frac{k\pi t}{2}\right),\hspace{0,2cm}com\hspace{0,2cm}k\in\Z^*.}
\end{equation*}
