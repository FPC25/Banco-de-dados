\linespread{1.5}

\textbf{Solução}

\textbf{a)} Podemos escrever $\pi^t$ como $\pi^t = e^{ln(\pi)^t}$

A imagem desse original pode ser calculada da seguinte forma:
\begin{equation*}
    F(s) = \int_0^\infty f(t)e^{-st} dt = \int_0^\infty e^{ln(\pi)^t}e^{-st} dt = \int_0^\infty e^{(ln\pi - s)t}dt = \left[\frac{1}{ln\pi - s}e^{(ln\pi - s)t} - \right]^\infty_0
\end{equation*}

Como $Re(s) > ln\pi$ então $Re(s) - ln\pi > 0$ $\Longrightarrow$ $ ln\pi - Re(s) < 0$, logo:

\begin{equation*}
    \lim_{t->\infty}\cancel{\frac{1}{ln\pi - s}e^{(ln\pi - s)t}} - \frac{1}{ln\pi - s}e^{(ln\pi - s)0} 
\end{equation*}

\begin{equation*}
    \boxed{F(s) = \frac{1}{ln\pi - s}}
\end{equation*}, onde $F(s)$ é analítica em toda a região  $Re(s) > ln\pi$

\textbf{b)}

Utilizando a propriedade da derivação da imagem, temos que:
\begin{equation*}
    \begin{cases}
    (-1)^nt^nf(t) \xrightarrow{\mathcal{L}} F^{(n)}(s), \hspace{0.2cm} onde \hspace{0.2cm} f(t) = sin(wt) \hspace{0.2cm} Então\\
    (-1)^2t^2sin(wt) \xrightarrow{\mathcal{L}} F''(s),
    \end{cases}
\end{equation*}
\begin{equation*}
    sin(wt) \xrightarrow{\mathcal{L}} \frac{w}{(s^2+w^2)} \longrightarrow F''(s) =\left[\frac{w}{(s^2+w^2}\right]'' = \frac{-2w[(s^2+w^2)^2] + 8s^2w(s^2+w^2)}{(s^2+w^2)^4} = \boxed{\frac{6s^2w - 2w^3}{(s^2+w^2)^3}} 
\end{equation*}

\textbf{c)} A função $f(t)$ pode ser escrita como uma soma de funções de heavyside:
\begin{equation*}
    f(t) = h(t) - 2h(t-5) + h(t-10)
\end{equation*}

Utilizando a propriedade da linearidade, temos:
\begin{equation*}
    f(t) \xrightarrow{\mathcal{L}}F(s) = \mathcal{L}{h(t)} +  \mathcal{L}{-2h(t-5)} +  \mathcal{L}{h(t-10)} \longrightarrow \boxed{F(s) = \frac{1}{s} - \frac{2e^{-5s}}{s} + \frac{e^{-5s}}{s}}
\end{equation*}

\textbf{d)} 

Nesta questão, podemos utilizar a propriedade da Integração de Imagem:

\begin{equation*}
    \frac{f(t)}{t} \xrightarrow{\mathcal{L}} \int_s^\infty F(z)dz
\end{equation*}

Precisamos primeiro acahar a integral de $sinh(t)$, lembrando que $sinh(t) = \frac{e^t - e^{-t}}{2}$ $\therefore$

\begin{equation*}
    F(s) = \int_0^\infty \frac{e^t - e^{-t}}{2} e^{-st} dt = \int_0^\infty \frac{e^{(1-s)t}}{2}dt - \int_0^\infty \frac{e^{-(1+s)t}}{2} = 
    \left[\frac{e^{(1-s)t}}{2(1-s)}\right]^\infty_0 - \left[\frac{e^{-(1+s)t}}{-2(1-s)}\right]^\infty_0 = 
\end{equation*}
\begin{equation*}
    = \frac{1}{2(s-1)} - \frac{1}{(2(1+s)} = \frac{(s+1) - (s-1)}{2(s-1)(s+1)} = \frac{1}{s^2 - 1}
\end{equation*}
Então
\begin{equation*}
    \frac{sinh(t)}{t} \xrightarrow{\mathcal{L}} \int_s^\infty F(z)dz = \int_s^\infty \frac{1}{z^2-1}dz = \int_s^\infty \frac{1}{(z-1)(z+1)}dz
\end{equation*}
Por frações parciais, temos
\begin{equation*}
    \int_s^\infty F(z)dz = \frac{1}{2} \left[\int_s^\infty -\frac{1}{(z+1)}dz +  \int_s^\infty \frac{1}{z-1}dz\right] = \left[\frac{-1}{2}ln|z+1| + \frac{1}{2}ln|z-1|\right]^\infty_s = 
\end{equation*}
\begin{equation*}
    = \lim_{z->\infty} \left[\frac{-1}{2}ln|z+1| + \frac{1}{2}ln|z-1|\right] - \left[\frac{1}{2}ln\frac{s-1}{s+1}\right] = \lim_{z->\infty} \left[\frac{1}{2}ln\frac{1-\cancel{\frac{1}{z}}}{1+\cancel{\frac{1}{z}}}\right] - \frac{1}{2}ln\frac{s-1}{s+1}
\end{equation*}
\begin{equation*}
    \boxed{F(s) = \frac{1}{2}\left|\frac{s+1}{s-1}\right|}
\end{equation*}

\textbf{e)}

Considera $f(t) = t - (t-3)h(t-3)$

sabemos que:
\begin{equation*}
    \begin{cases}
    t\xrightarrow{\mathcal{L}}\frac{1}{s^2}\\
    (t-3)h(t-3) \xrightarrow{\mathcal{L}} \frac{e^{-3s}}{s^2}
    \end{cases}\therefore \boxed{F(s) = \frac{1}{s^2} - \frac{e^{-3s}}{s^2} = \frac{1-e^{-3s}}{s^2}}
\end{equation*}

\textbf{f)}
Lembremos que $cos(t) = \frac{e^{it} + e^{-it}}{2}$, então $cos^2(t) = \frac{1}{2} + \frac{1}{2}\left(\frac{e^{2it} + e^{-2it}}{2}\right) = \frac{1}{2} + \frac{1}{2}cos(2t)$, lembrando também que $cos(wt) \xrightarrow{\mathcal{L}} \frac{s}{s^2+w^2}$

\begin{equation*}
    \begin{cases}
    \frac{cos(2t)}{2} \xrightarrow{\mathcal{L}} \frac{s}{2(s^2+4)}\\
    \frac{h(t)}{2} \xrightarrow{\mathcal{L}}\frac{1}{2s}
    \end{cases}\therefore \boxes{F(s) = \frac{1}{2}\left(\frac{s}{s^2+4} + \frac{1}{s}\right) 
\end{equation*}

\textbf{f)}
Aqui podemos aplicar a propriedade da mudança de origem da imagem:
\begin{equation*}
    f(t)\xrightarrow{\mathcal{L}} F(s) = \int_0^\infty e^{at}cos(wz)e^{-sz}dz = \frac{1}{(a-s)^2+w^2}\left\{\left[(a-s)cos(wt)e^{(a-s)z}\right]^\infty_0 + \left[wsin(wt)e^{(a-s)t}\right]_0^\infty\right\} = 
\end{equation*}
\begin{equation*}
    \frac{1}{(a-s)^2+w^2}(s-a)\therefore \boxed{F(s) = \frac{s-a}{(s-a)^2 + w^2}}
\end{equation*}

\textbf{h)} 

Para achar a imagem podemos utilizar a propriedade da derivação de imagens:
\begin{equation*}
    (-1)^nt^ng(t) \xrightarrow{\mathcal{L}} G^{(n)} = F(s)
\end{equation*}

tomando então $g(t) = e^{t-3}$, então:
\begin{equation*}
    (-1)^2t^2e^{t-3} \xrightarrow{\mathcal{L}} G'' = F(s) = \int_0^\infty e^{t-3-st}dt = e^{-3} \int_0^\infty e^{(1-s)t}dt = e^{-3}\left[\frac{e^{(1-s)t}}{(1-s)}\right]^\infty_0 = \frac{e^{-3}}{1-s}\therefore
\end{equation*}
\begin{equation*}
    G''(s) = \frac{\partial}{\partial s}\left[\frac{e^{-3}}{(1-s)}\right] = \frac{2e^{-3}}{(s-1)^3}
\end{equation*}
\begin{equation*}
    \therefore \boxed{F(s) = \frac{2e^{-3}}{(s-1)^3}} 
\end{equation*}

\textbf{i)}

Usamos da propriedade da mudança de origem da imagem:
\begin{equation*}
    e^{at}g(t) \xrightarrow{\mathcal{L}} G(s-a) = F(s)
\end{equation*}
onde $a=-3$ e $g(t) = sin(5t)$.

Sabemos que:
\begin{equation*}
     sin(wt) \xrightarrow{\mathcal{L}} \frac{w}{s^2+w^2}
\end{equation*}
então 
\begin{equation*}
     sin(5t) \xrightarrow{\mathcal{L}} \frac{5}{s^2+25} \therefore
\end{equation*}

\begin{equation*}
    \boxed{F(s) = G(s+3) = \frac{5}{(s+3)^2+25}}
\end{equation*}

\textbf{j)}

Poderíamos tentar achar a imagem utilizando da linearidade e a da propriedade da derivada do original, todavia, nas definições da transformada, se pede que a função $f(t)\in\R$, mas como não é o caso, podemos dizer que essa $f(t)$ não tem transformada de Laplace.

\textbf{k)}

Neste caso usamos da propriedade do atraso de originais:

\begin{equation*}
    \begin{cases}
    cosh(t-3) \xrightarrow{\mathcal{L}} e^{-3s}G(s)\\
    cosh(t)\xrightarrow{\mathcal{L}} 
    \end{cases}\longrightarrow \boxed{F(s) = \frac{s}{s^2-1}e^{-3s}}
\end{equation*}
