\linespread{1.5}

\textbf{Solução:}

Sendo a frequência angular fundamental definida por {\leftarge $w=\frac{2\pi}{T}$}, onde podemos reescrevê-la de tal forma que isolamos o período fundamental:
{\leftarge \begin{equation*}
    \label{eq:per_fund}
    T = \frac{2\pi}{w}
\end{equation*}}
Dado que a forma geral de senos e cossenos é dado como:
\begin{equation*}
    \begin{cases}
        \cos{(wx)}\\
        \sin{(wx)}
    \end{cases}
\end{equation*}

\textbf{a)}
Usando as equações acima, obtemos que os períodos fundamentais para:
\begin{enumerate}[i]
    \item $\cos{(x)}$ e $\sin{(x)}$, temos que $w=1$, logo $T=\frac{2\pi}{1} = 2\pi$ para ambos os casos;
    \item $\cos{(2x)}$, temos que $w=2$, logo $T=\frac{2\pi}{2} = \pi$;
    \item $\sin{(\pi x)}$, temos que $w=\pi$, logo $T=\frac{2\pi}{\pi} = 2$;
    \item $\sin{(2\pi x)}$, temos que $w=2\pi$, logo $T=\frac{2\pi}{2\pi} = 1$.
\end{enumerate}

\textbf{b)} Seguindo o mesmo raciocínio do item \textbf{a)}:

\begin{enumerate}[i]
    \item $\cos{(nx)}$, temos que $w=n$, logo $T=\frac{2\pi}{n}$;
    \item $\cos{\left(\frac{2\pi x}{k}\right)}$, temos que $w=\frac{2\pi}{k}$, logo $T=\frac{2\pi}{w} = \frac{2\pi}{2\pi}k = k$;
    \item $\cos{\left(\frac{2n\pi x}{k}\right)}$, temos que $w=\frac{2n\pi}{k}$, logo $T=\frac{2\pi}{w} = \frac{2\pi}{2n\pi}k = \frac{k}{n}$;
\end{enumerate}