\linespread{1.5}

\textbf{\textit{Solução}}
Como é uma E.D.O. de segunda ordem de coeficientes constantes não homogênea  teremos que utilizar o método de Lagrange. Sabemos que:
\begin{equation*}
    y_G(t) = y_H(t) + y_P(t)
\end{equation*}
onde 
\begin{itemize}
    \item $y_G(t)$ é a solução geral da equação;
    \item $y_H(t)$ é a solução da homogênea ;
    \item $y_P(t)$ é a solução particular.
\end{itemize}

Resolvemos então primeiramente a homogênea:
\begin{equation*}
    y''(t) - y'(t) - 6y(t) = 0
\end{equation*}
Temos o seguinte polinômio característico:
\begin{equation*}
    p^2 - p - 6p = 0 \rightarrow p = \frac{1 \pm 5}{2}
\end{equation*}
Nos dando que:
\begin{equation*}
    \begin{cases}
        p_1 = 3\\
        p_2 = -2
    \end{cases}
\end{equation*}
então:
\begin{equation*}
    \begin{cases}
        y_1 = C_1e^{3t}\\
        y_2 = C_2e^{-2t}
    \end{cases}
\end{equation*}

então:
\begin{equation*}
    y_H = C_1e^{3t} + C_2e^{-2t}
\end{equation*}

Agora voltemos buscamos a solução da particular pelo método de Lagrange. Para isso descobrimos o determinante da matriz:
\begin{equation*}
    W = \begin{vmatrix} 
        y_1' & y_2'\\
        y_1  & y_2
        \end{vmatrix} = 
        \begin{vmatrix} 
        3C_1e^{3t} & -2C_2e^{-2t}\\
        C_1e^{3t} &  C_2e^{-2t}
        \end{vmatrix} = 3C_1C_2e^t - (-2)C_1C_2e^t = 5C_1C_2e^t
\end{equation*}

e agora determinamos $y_P$:
\begin{equation*}
  y_P = y_1u_1 + y_2u_2 = y_1\int \frac{r(t)y_2}{W} dt + y_2\int \frac{-r(t)y_1}{W} dt = (I) + (II)
\end{equation*}
onde $r(t) = 3e^{-t}$. Resolvendo $(I)$:
\begin{equation*}
    (I) = \cancel{C_1}e^{3t}\int \frac{3e^{-t}\cancel{C_2}e^{-2t}}{5\cancel{C_1}\cancel{C_2}e^t}dt = \frac{3e^{3t}}{5}\int e^{-4t}dt = \frac{3e^{3t}}{5}\left(\frac{-e^{-4t}}{4}\right) = \frac{-3e^{-t}}{20}
\end{equation*}
Resolvendo $(II)$:
\begin{equation*}
    (II) = \cancel{C_2}e^{-2t}\int \frac{-3e^{-t}\cancel{C_1}e^{3t}}{5\cancel{C_1}\cancel{C_2}e^t}dt = \frac{-3e^{-2t}}{5}\int e^tdt = \frac{-3e^{-2t}}{5}e^t = \frac{-3e^{-t}}{5}
\end{equation*}
Assim:
\begin{equation*}
    y_P = \frac{-3e^{-t}}{20} - \frac{3e^{-t}}{5} = -\frac{-3e^{-t}}{4}
\end{equation*}

Desta forma, quando somamos as soluções particular e homogênea obtemos que a geral é:
\begin{equation*}
    y_G = C_1e^{3t} + C_2e^{-2t} - \frac{-3e^{-t}}{4}
\end{equation*}