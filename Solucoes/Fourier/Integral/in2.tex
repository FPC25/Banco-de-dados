\linespread{1.5}

\textbf{Solução}

\textbf{a)} Integral de Fourier cosseno:

Como a função $f(x)$ satisfaz os critérios do teorema de Fourier, então a integral de Fourier cosseno de $f(x)$ é dada por:

\begin{equation*}
    f(x) =  \int_0^\infty A(w) cos(wx)dw
\end{equation*}
onde:
\begin{equation*}
    A(w) = \frac{2}{\pi}\int_0^\infty f(v) cos(wv)dv
\end{equation*}

Dessa forma:
\begin{equation*}
    A(w) = \frac{2}{\pi} \int_0^a cos(wv)dv = \frac{2}{\pi}\left[\frac{sin(wv)}{w}\right]^a_0 = \frac{2sin(wa)}{w\pi}
\end{equation*}
logo:
\begin{equation*}
    \boxed{ \frac{2}{\pi}\int_0^\infty \frac{sin(wa)cos(wx)}{w}dw }
\end{equation*}

\textbf{b)} integral de Fourier seno

Já foi dito que a função $f(x)$ satisfaz os critérios de Fourier, logo a integral de Fourier seno é dada por:
\begin{equation*}
    f(x) = \int_0^\infty B(w)sin(wx)dw
\end{equation*}
onde:
\begin{equation*}
    B(w) = \frac{2}{\pi} \int_0^\infty f(v) sin(wv)dv
\end{equation*}

Dessa forma teremos que:
\begin{equation*}
    B(w) = \frac{2}{\pi} \int_0^a sin(wv)dv = \frac{2}{\pi}\left[\frac{-cos(wv)}{w}\right]^a_0 = \frac{2}{\pi}\left[\frac{-cos(wa)+1}{w}\right]
\end{equation*}
\begin{equation*}
    \therefore \boxed{ f(x) = \frac{2}{\pi}\int_0^\infty \left[\frac{-cos(wa)+1}{w}\right] sin(wx)dw }
\end{equation*}