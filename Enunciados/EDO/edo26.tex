\linespread{1.5}
Quando um corpo se move através de um fluido viscoso sob ação de uma força \textbf{\textit{F}}, a força resultante é \textbf{$F-k\eta\nu$}, sendo \textit{\textbf{k}} dependente da forma do corpo, \textbf{$\nu$} a velocidade do corpo e \textbf{$\eta$} o coeficiente de viscosidade. Diante do cenário apresentado, obtenha a velocidade como função do tempo, supondo que o movimento seja retilíneo, que a força aplicada seja constante e que $\nu(0) = \nu_0$.